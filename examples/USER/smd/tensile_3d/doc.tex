\documentclass[11pt]{scrartcl}

\usepackage{ucs}
\usepackage[utf8x]{inputenc}
\usepackage[english]{}
\usepackage{amsmath,amssymb,amstext}
\usepackage{graphicx}
\usepackage[automark]{scrpage2}

\usepackage{color}

\pagestyle{scrheadings}

\begin{document}

\subsection{Bar under Tension}
\label{sec:3d-bar}

This example serves to illustrate the features of Total-Lagrange SPH in comparison to traditional FEM
simulations. A 3D bar is elongated by pulling it apart from both sides at high velocity. Due to
Poisson's deformation ratio the cross section in the middle of the bar is reduced and a non-homogeneous
stress distribution is developed. Since a damage model is implemented, after reaching certain
deformation, failure will occure at the middle of the bar. Following the failure, the DCB will be
reversed till a huge deformation by compression is reached. The LAMMPS input script is found within
the directory \texttt{examples/USER/smd/tensile\_3}.
\\\\
The script creates a lattice of particles, applies corresponding boundary conditions, invokes the
required pair style and time integration fix 'commnads' and specify output quanities for the
visualization. Below it is explained how the simulation setup is organized in the scrip.

\subsubsection{Load Geometry}
The script starts difining a series of constant parameters relevant for the simulation. A consistent
unit system is used by means of the command 'units si', in this case millimeters (mm), milliseconds
(ms), kilograms (kg) and Giga Pascal (GPa).The geometrical discretization is performed using an FEM
tool and stored in a data file which is read in the script to generate the lattice of particles and
intialize the simulation box and particle properties. The SPH kernel radius is set to a value,such that
approximately 25 neighbor particles interact with each particle. Figure \ref{fig:fem-sph} shows the
SPH discretization and the original FEM mesh from wich the lattice of particles was based on.
\footnote{Discretisation using LS-PrePost}
\\
\begin{figure}[htbp]
  \begin{minipage}{0.495\textwidth}
    \centering
      \includegraphics[width=1.00\textwidth]{./frontview_fem.jpg}
      \includegraphics[width=1.00\textwidth]{./isoview_fem.jpg}
  \end{minipage}\hfill
  \begin{minipage}{0.495\textwidth}
    \centering
      \includegraphics[width=1.00\textwidth]{./frontview_sph.png}
      \includegraphics[width=1.00\textwidth]{./isoview_sph.png}
  \end{minipage}
  \caption{FEM discretization (left) and SPH discretization (right).}
  \label{fig:fem-sph}
\end{figure}

To apply the boundary conditions, an independent group is defined for each end of bar to contain the
last layer of particles. Another group is defined including the whole lattice of particles, to specify
the pair style which determines the material model.

\subsubsection{Material properties and pair style/interaction}
A Total-Lagrange \texttt{pair\_style tlsph} is used to modell the material. The \texttt{*COMMON}
keword is mandatory and defines general material properties, which are not related with a specific
material model, such as density, Young's modulus and Poisson's ratio. The parameters Q1 and Q2 define
the linear and cuadratic parts of the artificial viscosity. The parameter hg defines the 'hourglas
control' coefficient, which should be chosen between 10 and 100. The Cp parameter, which defines the
specific heat capacity, is the last argument of this keyword. The second keyword activates a linear
materal model, Hookean linear elasticity, for the deviatoric part of the deformation and the third
keyword activates an EOS for the volumetric deformation.
\\\\
Additionally, a herzian potential pair interaction (\texttt{smd/herz}) is defined to avoid
interpenetration during the compression once failure is alredy occured. To allow both pair
interactions (tlsph and contact) the \texttt{pair\_style} command has to be used with the style
\texttt{hybrid/overlay}.

\subsubsection{Set Boundary conditions}
Dirichlet Boundary Conditions (DCB) are applied by setting the acceleration/force to zero on the groups
mention above and a constant velocity with the same magnitude on both groups but opposite direcctions.

\subsubsection{Time Integration}
The \texttt{fix smd/integrate\_tlsph} command is used to define the time integration of the system.
A stable time increment for the explicit Velocity-Verlet integration approach used by \texttt{LAMMPS}
is computed with the \texttt{fix smd/adjust\_dt} command and applying a factor of 0.1 recomended for
SPH algorithms.

\subsubsection{Post-process and outputs}
\texttt{compute} and \texttt{variable} commands are used to calculate relevant quantities which will
be written in the \texttt{dump} file, such as stresses, strains and deformation gradient, just to
mention some examples.

\textcolor{green}{Green Text.}
\color{black}

\begin{figure}
  \begin{center}
    \includegraphics[width=0.8\textwidth]{isoview_fem.jpg}
    \includegraphics[width=0.8\textwidth]{dt_01_sph.png}
  \caption{Plastic Strain}
  \label{fig:plast-strain}
  \end{center}
\end{figure}

\newpage

In section \ref{sec:3d-bar} ...

Discretization of the bar using exahedra finite elements..

\end{document}
